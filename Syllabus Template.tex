\documentclass{article}

\usepackage{styles}
\usepackage{titlesec}

\PaperTitle{Syllabus - Course}
\begin{document}
\maketitle

\begin{table}[h]
\centering
\begin{tabular}{lll} 
\hline
\textbf{Name} & 
\textbf{Email: \href{mailto:email address}{Email Address}} & 
\textbf{Office: Office} \\
\multicolumn{3}{c}{Office Hours: TBD}\\
\hline                                   
\end{tabular}
\end{table}

\section*{Welcome to (Class Name)!}

Introduction and Welcome Text

\subsection*{Course Meeting Times \& Places}

\begin{table}[ht]
\centering
\begin{tabular}{ccccll}
\cline{1-4}
\multicolumn{1}{|c|}
{\textbf{Course}} & \multicolumn{1}{c|}
{\textbf{Days}} & \multicolumn{1}{c|}
{\textbf{Time}}      & \multicolumn{1}{c|}
{\textbf{Room}} &  &  \\ \cline{1-4}

%Each section below is one line in the table.  Copy and Paste entire section to add new lines.

\multicolumn{1}{|c|}
{Course Number}            & \multicolumn{1}{c|}
{Days of the week}         & \multicolumn{1}{c|}
{Times} & \multicolumn{1}{c|}
{Rooms}		&  &  \\ \cline{1-4}

\multicolumn{1}{|c|}
{Course Number}            & \multicolumn{1}{c|}
{Days of the week}         & \multicolumn{1}{c|}
{Times} & \multicolumn{1}{c|}
{Rooms}		&  &  \\ \cline{1-4}


\end{tabular}
\end{table}


%One Column inside box, standard usage

\begin{tcolorbox}[width=\textwidth,colback={white},
title={\textbf{Course Materials}},
colbacktitle=color2!10,
coltitle=color1]    
Materials preamble
	\begin{enumerate}[noitemsep]
		\item{Textbook Information}
	\end{enumerate}
\end{tcolorbox}  

%Two Columns inside box
\begin{tcolorbox}[width=\textwidth,colback={white},
title={\textbf{Course Materials}},
colbacktitle=color2!10,
coltitle=color1] 

Materials preamble, this will stretch across the two columns below

\begin{multicols}{2}

\textbf{Lecture Materials.}
	\begin{enumerate}[noitemsep]
   		\item{Textbook: Textbook information}
	\end{enumerate}
\columnbreak
\textbf{Lab Materials.}
	\begin{enumerate}[noitemsep]
   		\item{Lab Materials}
	\end{enumerate}

\end{multicols}
\end{tcolorbox} 


\section*{Course Policies}

\subsection*{Ground Rules for the Learning Environment}
\begin{itemize}[noitemsep]
	\item{\textbf{Be Respectful.}  This is an inclusive learning environment, with a range of experiences with Chemistry present.  No one was born a master of chemistry, and we all strive to continue to learn together.}
	\item{\textbf{Be engaged!} Take notes, ask questions. There will be several avenues to ask questions beyond raising your hand (which can be daunting by itself), so please use LabPal and class time to work towards understanding the material! I welcome more specialized questions, but may push them to Office Hours if they fall outside the scope of our course.}
	\item{\textbf{Use course resources effectively!} Frequently check the Course site and use the activities posted to practice and get feedback.}
\end{itemize}

\subsection*{Communications Policy}
This course covers a lot of content in a single quarter, and I want to provide clear avenues of interacting both with myself and  your peers throughout the term.  Here are some guidelines to help you get help when you need it. E-mail (\href{mailto:Email Address}{Email Address}) is the best way to get ahold of me!

\begin{itemize}[noitemsep]
	\item{I may answer your question in an announcement to the class if it is something that I assume many of your peers are also wondering or an important clarification. Keep an eye on announcements!}
	\item{Any messages received after 4 PM will be addressed the following morning.}
	\item{Any messages received after 4 PM on Friday will be addressed on the following Monday morning.}
	\item{Use the LabPal forum to ask content related questions, \textbf{not e-mail}. I am more likely to respond on the forum where your peers can get the same help.}
\end{itemize}

\subsection*{Absence Policy}
\paragraph{PLEASE DO NOT COME TO CLASS IF YOU ARE SICK.} If you are ill, have anything other than a green campus pass, or believe you have been recently been exposed to COVID-19, please notify me immediately.  You are accountable for getting notes and making up missed work!  Keep in contact with your peers and lab partner.

\paragraph{Repeated unexcused absences will be grounds for a failing grade of the course.} Unexcused absences will result in lost points, as there are no makeups for missed lab reports and other graded activities.
If you have a planned absence for any excusable reason (athletic competition, family emergency, religious holiday, etc), inform me at \textbf{least 1 week before} the class you plan to miss.

\section*{Academic Integrity}
The University Code of Academic Integrity is central to the ideals of this course. Students are expected to be independently familiar with the Code and to recognize that their work in the course is to be their own original work that \emph{truthfully represents the time and effort applied.} Violations of the Code are most serious and will be handled in a manner that fully represents the extent of the Code and that befits the seriousness of its violation.

Any form of falsely claiming work to be your own when it was not - such as \emph{using unauthorized aids on an exam, falsifying lab data, using uncited data, or plagiarizing another person’s writing} — is considered violations of academic integrity. This applies to all materials generated for the course, both in the laboratory and the classroom.
More generally, please familiarize yourself with Cal Poly's Code of Academic Integrity, which applies to this course. In the  event that any concerns do arise on this score, I will forward all related materials to Office of Student Rights \& Responsibilities.  Following the guidelines established by the Academic Senate, \textbf{a Failing Course Grade may be applied for any instance of academic dishonesty.}


\section*{Campus Resources For Students}

\subsection*{Basic Needs Support}
If you face challenges securing food, housing or other basic needs, you are not alone, and Cal Poly can help during this time of crisis. We invite you to learn about the many resources available to support you through Cal Poly's Basic Needs initiative at \href{basicneeds.calpoly.edu}{basicneeds.calpoly.edu} or by contacting \href{mailto:deanofstudents@calpoly.edu}{deanofstudents@calpoly.edu}. An extensive list of critical care resources is also listed and updated on the Cal Poly Coronavirus information pages \href{https://coronavirus.calpoly.edu/student-care-resources}{Student Care Resources page}.

\subsection*{Students with Disabilities}
Persons who wish to request disability-related accommodations should contact the Disability Resource Center in Building 124, Room 119. Phone: (805) 756-1395 or (805) 756-6266 or website: \href{http://www.drc.calpoly.edu/}{Disability Resource Center - Cal Poly}. Office hours are Monday-Friday from 8:00 AM – 4:30 PM. Some accommodations may take up to several weeks to arrange. If you are a student with a disability, please consider discussing your needs and possible accommodations with me as soon as possible.

\subsection*{Corona Virus Support}
If you have specific questions about the services available to you through Cal Poly during the coronavirus outbreak, please visit the \href{http://coronavirus.calpoly.edu/}{Cal Poly Coronavirus website}. For specific information on frequently asked questions about Cal Poly classes and programs, visit the \href{https://coronavirus.calpoly.edu/classes-and-programs}{Classes and Programs section} of this website.
\paragraph{Note:} You will be required to show your campus pass status upon entering the classroom each day!  Please be prepared!

\subsection*{Mental Health Support}
Recent nationwide surveys of college students consistently find that stress, sleep problems, anxiety, depression, interpersonal concerns, death of a significant other and alcohol use are among the top ten health impediments to academic performance. Students experiencing personal problems or situational crises are encouraged to contact Cal Poly's Counseling Services (805-756-2511) for assistance, support and advocacy. This service is free and confidential.

\textbf{Please contact me if you need help accessing or using any of the above resources!}

\pagebreak

\section*{Activities, Assessments, and Grades}

\setlength{\intextsep}{-10pt}%
\setlength{\columnsep}{10pt}%

\begin{wraptable}{hr}{0.3\textwidth}
	\begin{tabular}{ccc}\\
		\textbf{Activity} & \textbf{Value (\%)}  \\ \hline
		Quizzes & 25 \\ \hline
		Graded Activities & 20\\ \hline
		Midterm Exams& 30\\ \hline
		Final Exam & 25 \\ \hline
		& \\ 
		\textbf{Grade} & \textbf{\% Range}  \\ \hline
		A to A- & 90-100 \\ \hline
		B+ to B- & 80-90 \\ \hline
		C+ to C- & 70-80 \\ \hline
		D+ to D- & 60-70 \\ \hline
		F & Below 60 \\ \hline
	\end{tabular}
\end{wraptable}

\paragraph{Activities.} While the laboratory is a great place to get your hands "dirty" (while wearing gloves of coures!) the classroom is a place to engage with ideas. These in-class activities can take the form of group problem sets, literature discussions, and informal checks on understanding.


\paragraph{Assessments}
\paragraph{Quizzes} Our course will employ weekly quizzes, unless there is another notable assessment scheduled.  All quizzes are cumulative, and may differ in length and intensity.  Regularly working through problem sets and reviewing material will ensure that you can be successful on these regular assessments.

\textbf{Expect a Quiz Every Friday during the Term!}

\paragraph{Midterms.} We will also have two midterms throughout the term.  These examinations will be more formal than a quiz and will take the entirety of the class period (50 minutes).  These exams will also be cumulative, but will generally focus on material since the previous exam.

\paragraph{Final Exam.} Pay attention to the final exam schedule and plan ahead.  Make travel plans accordingly, as finals will not be rescheduled without a university approved cause. 

\textbf{Course Number: Final Exam Date and Time}.  

The final exam is Cumulative and will take place in our normal classroom during Finals Week.

\paragraph{Note:}Grade ranges and weights are subject to change as the term proceeds.


\section*{Course Learning Objectives}
\emph{Disclaimer: Topics, Pacing, and Assessments are subject to change.}

At the end of the term, students will be able to...

\subsection*{Module 1: Title}
\begin{itemize}[noitemsep]

\item{LO 1.1}

\end{itemize}

\subsection*{Module 2: Title}
\begin{itemize}[noitemsep]
	
	\item{LO 2.1}
	
\end{itemize}

\section*{Course Schedule}
Schedule is tentative, the dates of each topic are subject to change.  However, we will cover all the course learning objectives by the end of the term.

\begin{table}[ht]
	\centering
\begin{tabular}{|c|l|l|}
\hline
\textbf{Week} & \textbf{Topic}     & \textbf{Assessments/Notes}               \\ \hline
1              & Wk 1 Topic            & Quiz 1                                                                     \\ \hline
2             & Wk 2 Topic            & Quiz 2                                                                     \\ \hline
3             & Wk 3 Topic            & Quiz 3                                                                     \\ \hline
4             & Wk 4 Topic            & Midterm 1                                                                  \\ \hline
5             & Wk 5 Topic            & Quiz 4                                                                     \\ \hline
6             & Wk 6 Topic            & Quiz 5                                                                     \\ \hline
7             & Wk 7 Topic            & Quiz 6                                                                     \\ \hline
8             & Wk 8 Topic            & Quiz 7                                                 \\ \hline
9             & Wk 9 Topic            & Midterm 1                                           \\ \hline
10           & Wk 10 Topic           & No Class Monday, Midterm 2           \\ \hline
\end{tabular}
\end{table}

\end{document}
